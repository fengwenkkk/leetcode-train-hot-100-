\subsubsection{内容简介}

包含了 LeetCode hot 100 题目

\chapter{算法复杂度}

「 大O符号表示法 」,即 T(n) = O(f(n))

\subsubsection{时间复杂度}

例子:

for(i=1; i<=n; ++i)
{
   j = i;
   j++;
}
通过「 大O符号表示法 」,这段代码的时间复杂度为:O(n) ,为什么呢?

在 大O符号表示法中,时间复杂度的公式是: T(n) = O( f(n) ),其中f(n) 表示每行代码执行次数之和,而 O 表示正比例关系,这个公式的全称是:算法的渐进时间复杂度。

我们继续看上面的例子,假设每行代码的执行时间都是一样的,我们用 1颗粒时间 来表示,那么这个例子的第一行耗时是1个颗粒时间,第三行的执行时间是 n个颗粒时间,第四行的执行时间也是 n个颗粒时间(第二行和第五行是符号,暂时忽略),那么总时间就是 1颗粒时间 + n颗粒时间 + n颗粒时间 ,即 (1+2n)个颗粒时间,即: T(n) = (1+2n)*颗粒时间,从这个结果可以看出,这个算法的耗时是随着n的变化而变化,因此,我们可以简化的将这个算法的时间复杂度表示为:T(n) = O(n)

为什么可以这么去简化呢,因为大O符号表示法并不是用于来真实代表算法的执行时间的,它是用来表示代码执行时间的增长变化趋势的。

所以上面的例子中,如果n无限大的时候,T(n) = time(1+2n)中的常量1就没有意义了,倍数2也意义不大。因此直接简化为T(n) = O(n) 就可以了。

对数阶O(logN)
还是先来看代码:

int i = 1;
while(i<n)
{
    i = i * 2;
}
从上面代码可以看到,在while循环里面,每次都将 i 乘以 2,乘完之后,i 距离 n 就越来越近了。我们试着求解一下,假设循环x次之后,i 就大于 2 了,此时这个循环就退出了,也就是说 2 的 x 次方等于 n,那么 x = log2^n
也就是说当循环 log2^n 次以后,这个代码就结束了。因此这个代码的时间复杂度为:O(logn)

## 空间复杂度

如果算法执行所需要的临时空间不随着某个变量n的大小而变化,即此算法空间复杂度为一个常量,可表示为 O(1)

int[] m = new int[n]
for(i=1; i<=n; ++i)
{
   j = i;
   j++;
}
这段代码中,第一行new了一个数组出来,这个数据占用的大小为n,这段代码的2-6行,虽然有循环,但没有再分配新的空间,因此,这段代码的空间复杂度主要看第一行即可,即 S(n) = O(n)
