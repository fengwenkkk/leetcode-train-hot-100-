\chapter{哈希表}

**解题关键**:匹配,怎么设置哈希表的键值


1.两数之和

2.分词

```

unordered_set< xxx/ xxx XXX > result_set;//建立

map.insert(XXX/ {xxx,XXX}); //放入

`find(key)`: 查找指定键的元素,返回指向该元素的迭代器或者指向 `end()` 的迭代器(如果未找到)。

`erase(key)`: 从哈希表中删除指定键的元素。

-   `size()`: 返回哈希表中元素的数量。
- -   `count(key)`: 返回指定键在哈希表中出现的次数(在集合中,要么是 0 要么是 1)。
-   `clear()`: 删除哈希表中的所有元素,使之成为空表。
- -   `empty()`: 如果哈希表为空,则返回 `true`,否则返回 `false`。

myMap[key] = value;

if (myMap.find(key) != myMap.end()) {
    // 查找键存在
}
```
1.两数之和
思路:1.使用数组映射,将元素作为key,下标作为value
2.对元素排序,寻找哈希表有无相同值,有-》返回下标组;无-》将该元素和下标加入哈希表

2.分词


3.最长连续序列
<!--stackedit_data:
eyJoaXN0b3J5IjpbMTg5Nzc2ODgxNSwxODc4MDExNzYzLC0xNj
UzMjczNDI1LDg4NzA4MjcwXX0=
-->